\chapter{Wissenschaftlicher Teil}
\section{Background Subtraction}
\subsection{Theorie}
%rewrite, taken from background sub opencv tutorial, look for correct source
Background subtraction is a major preprocessing step in many vision-based applications. For example, consider the case of a visitor counter where a static camera takes the number of visitors entering or leaving the room, or a traffic camera extracting information about the vehicles etc. In all these cases, first you need to extract the person or vehicles alone. Technically, you need to extract the moving foreground from static background. \\
If you have an image of background alone, like an image of the room without visitors, image of the road without vehicles etc, it is an easy job. Just subtract the new image from the background. You get the foreground objects alone. But in most of the cases, you may not have such an image, so we need to extract the background from whatever images we have. It becomes more complicated when there are shadows of the vehicles. Since shadows also move, simple subtraction will mark that also as foreground. It complicates things. \\
Several algorithms were introduced for this purpose. OpenCV has implemented three such algorithms which are very easy to use. We will see them one-by-one.

Weiterhin werden die drei beliebstesten Background Subtraction Methoden erläutert und verglichen. Es wird oft das Gaussian Mixture Model erwähnt, das sollte wahrscheinlich hier erklärt werden. Darauf basiert (bis jetzt) MOG und KNN
\subsection{BackgroundSubtractorMOG 1 and 2}
% all from paper
The simplest form of the reference image is a time-averaged background image. This method suffers from many problems and requires a training period absent of foreground objects. The motion of background objects after the training period and foreground objects motionless during the training period would be considered as permanent foreground objects. In addition, the approach cannot cope with gradual illumination changes in the scene. These problems lead to the requirement that any solution must constantly reestimate the background model. Many adaptive background-modelling methods have been proposed to deal with these slowly-changing stationary signals.

Das heißt wenn ständig neu berechnet wird ist das ein ONLINE Learning Algorithmus.
\cite{Bowden}
\subsection{BackgroundSubtractorKNN}
K-nearest-neighbours
Wichtig: dieses Paper vergleicht schon existierende Algorithms
\cite{ZIVKOVIC2006773}
\subsection{BackgroundSubtractorGMG}
Dieses Paper benutzt nicht Gaussians Mixture, sondern Bayes Rule. Ziemlich mathematisch, look at towards the end
\cite{6315174}
\subsection{Fazit/Begründung Auswahl}
\section{Blob Detection}
Parameter Testing
\section{Glättungsfunktion}