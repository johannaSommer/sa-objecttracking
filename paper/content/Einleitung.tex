%Specs: 
%Wissenschaftlicher Teil 10-15 Seiten
%Maximal 60 Seiten insgesamt
%OpenCV Tutorials dürfen zitiert werden
\chapter{Einleitung}
\section{Kontext}
!!!EINLEITUNG HIER!!

Das am meisten in professionellem Sport benutze und das wohl meist bekannte Technologie ist das sogenannte Hawk-Eye System von Hawkeye Technologies. Seit der Saison 2015/2016 wird es in der Bundesliga eingesetzt und erstmals im DFB Pokal Finale im Jahr 2015. Dort wird das System für strittige Torentscheidungen eingesetzt. \\
Auch in mittlerweile drei von vier Grand Slam Turnieren wird diese Technologie eingesetzt, um knappe Entscheidungen abzusichern - so zum Beispiel ob ein Ball im 'Aus' ist oder nicht. Seit Ende 2013 wird auch für Linien-Entscheidungen im Badminton das System von Hawkeye Technologies angwandt. \\
Bei Sportübertragungen von Sky Sports und BBC findet es ebenfalls Anwendung, dort für Animation der vergangenen Spielzüge und ergänzende statistische Informationen \cite{hawkeye}.

Ein Augenmerk auf die statistische Analyse legt seit 2018 auch IBM und stellt für die Grand Slam Turniere den IBM SlamTracker bereit. Das Tool greift auf 12 Jahre Turnierdaten zurück und bietet Real-Time Statistiken für alle Live Spiele. Besonders interessant ist dabei z.B. das 'Momentum-Feature', das zeigt, welcher Spieler momentan einen Vorteil besitzt, basierend auf vorangegangenen Begegnungen und individuellen Stärken \cite{watson}.

Bis heute ist die Analyse menschlicher Bewegungen im Sport nur durch Videoanalyse und magnetischer Nachverfolgung möglich. Dies ist jedoch meist nur in genau preparierten Turniersituationen möglich und nicht für Amateursportler zugänglich. Nach Fortschritten in Micro-Elektro-Mechanischen-Systemen, auch MEMS Technologien genannt, eröffnen sich nun auch immer mehr Möglichkeiten in der Benutzung von am Körper befestigten Sensoren für Bewegungsanalyse im Sport \cite{sensors}. \\

Mit Aufnahmemöglichkeiten bald auch zugänglich für Amateursportler und den Möglichkeiten von KI für die Analyse spielt Technologie auch für traditionelle Sportarten eine immer wichtigere Rolle.   

\section{Aufgabenstellung}
genaue Aufgabenstellung, Abgrenzung der nicht erforderten Funktionalität, evtl. mit Herr Trost absprechen
\section{Voraussetzungen}
\subsection{Badminton}
Terminology	
\subsection{Hardware}
Kamerainfo
\subsection{Positionierung}
%hier Grafik machen mit Badmintonfeld + Kamera
Badmintonfeld: 
Breite Feld Einzel: 5.16m
Breite Feld Doppel: 6.1m
Länge Feld: 13.4m
Netzhöhe: 0.75m

Position der Kamera:
Hinten:
Auf Höhe der Mittellinie
2.8m entfernt von der hinteren Feldlinie
Höhe 1.35m

Seite:
ausgerichtet an dem Netz
3m von Doppellinie
Höhe 1.6m