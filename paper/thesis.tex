\documentclass[11pt,a4paper,oldfontcommands]{memoir}
\usepackage[utf8]{inputenc}
\usepackage[T1]{fontenc}
\usepackage{microtype}
\usepackage[dvips]{graphicx}
\usepackage{xcolor}
\usepackage{times}

\usepackage[
breaklinks=true,colorlinks=true,
%linkcolor=blue,urlcolor=blue,citecolor=blue,% PDF VIEW
linkcolor=black,urlcolor=black,citecolor=black,% PRINT
bookmarks=true,bookmarksopenlevel=2]{hyperref}

\usepackage{geometry}
% PDF VIEW
% \geometry{total={210mm,297mm},
% left=25mm,right=25mm,%
% bindingoffset=0mm, top=25mm,bottom=25mm}
% PRINT
\geometry{total={210mm,297mm},
left=20mm,right=20mm,
bindingoffset=10mm, top=25mm,bottom=25mm}

\OnehalfSpacing
%\linespread{1.3}

%%% CHAPTER'S STYLE
\chapterstyle{bianchi}
%\chapterstyle{ger}
%\chapterstyle{madsen}
%\chapterstyle{ell}
%%% STYLE OF SECTIONS, SUBSECTIONS, AND SUBSUBSECTIONS
\setsecheadstyle{\Large\bfseries\sffamily\raggedright}
\setsubsecheadstyle{\large\bfseries\sffamily\raggedright}
\setsubsubsecheadstyle{\bfseries\sffamily\raggedright}


%%% STYLE OF PAGES NUMBERING
%\pagestyle{companion}\nouppercaseheads 
%\pagestyle{headings}
%\pagestyle{Ruled}
\pagestyle{plain}
\makepagestyle{plain}
\makeevenfoot{plain}{\thepage}{}{}
\makeoddfoot{plain}{}{}{\thepage}
\makeevenhead{plain}{}{}{}
\makeoddhead{plain}{}{}{}


\maxsecnumdepth{subsection} % chapters, sections, and subsections are numbered
\maxtocdepth{subsection} % chapters, sections, and subsections are in the Table of Contents


%%%---%%%---%%%---%%%---%%%---%%%---%%%---%%%---%%%---%%%---%%%---%%%---%%%

\begin{document}

%%%---%%%---%%%---%%%---%%%---%%%---%%%---%%%---%%%---%%%---%%%---%%%---%%%
%   TITLEPAGE
%
%   due to variety of titlepage schemes it is probably better to make titlepage manually
%
%%%---%%%---%%%---%%%---%%%---%%%---%%%---%%%---%%%---%%%---%%%---%%%---%%%
\thispagestyle{empty}

{%%%
\sffamily
\centering
\Large

~\vspace{\fill}

{\huge 
Studienarbeit Dokumentation
}

\vspace{2.5cm}

{\LARGE
Johanna Sommer
}

\vspace{3.5cm}

Fakultät Informatik \\
DHBW Stuttgart

\vspace{3.5cm}

Betreuer: Sebastian Trost

\vspace{\fill}

Juni 2019

%%%
}%%%

\cleardoublepage
%%%---%%%---%%%---%%%---%%%---%%%---%%%---%%%---%%%---%%%---%%%---%%%---%%%
%%%---%%%---%%%---%%%---%%%---%%%---%%%---%%%---%%%---%%%---%%%---%%%---%%%

\tableofcontents

\clearpage

%%%---%%%---%%%---%%%---%%%---%%%---%%%---%%%---%%%---%%%---%%%---%%%---%%%
%%%---%%%---%%%---%%%---%%%---%%%---%%%---%%%---%%%---%%%---%%%---%%%---%%%

\chapter{Einleitung}
Specs: 
Wissenschaftlicher Teil 10-15 Seiten
Maximal 60 Seiten insgesamt
OpenCV Tutorials dürfen zitiert werden
\section{Kontext}
Badminton, Hawkeye System, aktuelle Relevanz, vielleicht ausblick auf kommende Sensorik im Sport
\section{Aufgabenstellung}
genaue Aufgabenstellung, Abgrenzung der nicht erforderten Funktionalität, evtl. mit Herr Trost absprechen
\section{Voraussetzungen}
\subsection{Hardware}
Kamerainfo
\subsection{Positionierung}
%hier Grafik machen mit Badmintonfeld + Kamera
Badmintonfeld: 
Breite Feld Einzel: 5.16m
Breite Feld Doppel: 6.1m
Länge Feld: 13.4m
Netzhöhe: 0.75m

Position der Kamera:
Hinten:
Auf Höhe der Mittellinie
2.8m entfernt von der hinteren Feldlinie
Höhe 1.35m

Seite:
ausgerichtet an dem Netz
3m von Doppellinie
Höhe 1.6m

\chapter{Wissenschaftlicher Teil}
\section{Background Subtraction}
Abgrenzung verschiedener Background Sub Methoden
\section{Blob Detection}
Parameter Testing
\section{Glättungsfunktion}

\chapter{Umsetzung}
\section{Framwork Auswahl}
Blender weil animationstool mit python script support und sogar api, so kann eine teilautomatisierte Pipeline geschaffen werden
\section{Design, Software Architektur}
\section{Testing}

\chapter{Schluss}
\section{Ergebnisse}
\section{Verbesserungen}
\section{Ausblick}



%todo fix bibliography
\bibliographystyle{unsrt}
\bibliography{sample}

\end{document}

